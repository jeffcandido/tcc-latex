%% abtex2-modelo-relatorio-tecnico.tex, v-1.9.7 laurocesar
%% Copyright 2012-2018 by abnTeX2 group at http://www.abntex.net.br/ 
%%
%% This work may be distributed and/or modified under the
%% conditions of the LaTeX Project Public License, either version 1.3
%% of this license or (at your option) any later version.
%% The latest version of this license is in
%%   http://www.latex-project.org/lppl.txt
%% and version 1.3 or later is part of all distributions of LaTeX
%% version 2005/12/01 or later.
%%
%% This work has the LPPL maintenance status `maintained'.
%% 
%% The Current Maintainer of this work is the abnTeX2 team, led
%% by Lauro César Araujo. Further information are available on 
%% http://www.abntex.net.br/
%%
%% This work consists of the files abntex2-modelo-relatorio-tecnico.tex,
%% abntex2-modelo-include-comandos and abntex2-modelo-references.bib
%%

% ------------------------------------------------------------------------
% ------------------------------------------------------------------------
% abnTeX2: Modelo de Relatório Técnico/Acadêmico em conformidade com 
% ABNT NBR 10719:2015 Informação e documentação - Relatório técnico e/ou
% científico - Apresentação
% ------------------------------------------------------------------------ 
% ------------------------------------------------------------------------

\documentclass[
  % -- opções da classe memoir --
  12pt,				% tamanho da fonte
  openright,			% capítulos começam em pág ímpar (insere página vazia caso preciso)
  twoside,			% para impressão em recto e verso. Oposto a oneside
  a4paper,			% tamanho do papel. 
  % -- opções da classe abntex2 --
  %chapter=TITLE,		% títulos de capítulos convertidos em letras maiúsculas
  %section=TITLE,		% títulos de seções convertidos em letras maiúsculas
  %subsection=TITLE,	% títulos de subseções convertidos em letras maiúsculas
  %subsubsection=TITLE,% títulos de subsubseções convertidos em letras maiúsculas
  % -- opções do pacote babel --
  english,			% idioma adicional para hifenização
  french,				% idioma adicional para hifenização
  spanish,			% idioma adicional para hifenização
  brazil,				% o último idioma é o principal do documento
  ]{abntex2}


% ---
% PACOTES
% ---
% ---- packs adicionais
\usepackage{lastpage}
\usepackage{xcolor}
\usepackage{listings}
\lstset{basicstyle=\ttfamily,
  showstringspaces=false,
  commentstyle=\color{red},
  keywordstyle=\color{blue}
}
\usepackage{hyperref}


% ---
% Pacotes fundamentais 
% ---
\usepackage{lmodern}			% Usa a fonte Latin Modern
\usepackage[T1]{fontenc}		% Selecao de codigos de fonte.
\usepackage[utf8]{inputenc}		% Codificacao do documento (conversão automática dos acentos)
\usepackage{indentfirst}		% Indenta o primeiro parágrafo de cada seção.
\usepackage{color}				% Controle das cores
\usepackage{graphicx}			% Inclusão de gráficos
\usepackage{microtype} 			% para melhorias de justificação
% ---

% ---
% Pacotes adicionais, usados no anexo do modelo de folha de identificação
% ---
\usepackage{multicol}
\usepackage{multirow}
% ---
  
% ---
% Pacotes adicionais, usados apenas no âmbito do Modelo Canônico do abnteX2
% ---
\usepackage{lipsum}				% para geração de dummy text
% ---

% ---
% Pacotes de citações
% ---
\usepackage[brazilian,hyperpageref]{backref}	 % Paginas com as citações na bibl
\usepackage[alf]{abntex2cite}	% Citações padrão ABNT

% --- 
% CONFIGURAÇÕES DE PACOTES
% --- 

% ---
% Configurações do pacote backref
% Usado sem a opção hyperpageref de backref
\renewcommand{\backrefpagesname}{Citado na(s) página(s):~}
% Texto padrão antes do número das páginas
\renewcommand{\backref}{}
% Define os textos da citação
\renewcommand*{\backrefalt}[4]{
  \ifcase #1 %
    Nenhuma citação no texto.%
  \or
    Citado na página #2.%
  \else
    Citado #1 vezes nas páginas #2.%
  \fi}%
% ---

% ---
% Informações de dados para CAPA e FOLHA DE ROSTO
% ---
\titulo{Implementações práticas de DSP e RF com GNURadio e HackRF}
\autor{Jefferson da Silva Cândido}
\local{Uberlândia, Minas Gerais}
\data{\the\year}
\instituicao{%
  Universidade Federal de Uberlândia -- UFU
  \par
  Faculdade de Engenharia Elétrica
  \par
  Graduação em Engenharia Eletrônica e de Telecomunicações}
\tipotrabalho{Trabalho de Conclusão de Curso}
% O preambulo deve conter o tipo do trabalho, o objetivo, 
% o nome da instituição e a área de concentração 
\orientador{Dr. Antônio Cláudio Paschoarelli Veiga}

\preambulo{Trabalho apresentado na Universidade Federal de Uberlândia como requisito para conclusão do curso de graduação em Engenharia Eletrônica e de Telecomunicações.}
% ---

% ---
% Configurações de aparência do PDF final

% alterando o aspecto da cor azul
\definecolor{blue}{RGB}{41,5,195}

% informações do PDF
\makeatletter
\hypersetup{
      %pagebackref=true,
    pdftitle={\@title}, 
    pdfauthor={\@author},
      pdfsubject={\imprimirpreambulo},
      pdfcreator={LaTeX with abnTeX2},
    pdfkeywords={abnt}{latex}{abntex}{abntex2}{relatório técnico}, 
    colorlinks=true,       		% false: boxed links; true: colored links
      linkcolor=blue,          	% color of internal links
      citecolor=blue,        		% color of links to bibliography
      filecolor=magenta,      		% color of file links
    urlcolor=blue,
    bookmarksdepth=4
}
\makeatother
% --- 

% --- 
% Espaçamentos entre linhas e parágrafos 
% --- 

% O tamanho do parágrafo é dado por:
\setlength{\parindent}{1.3cm}

% Controle do espaçamento entre um parágrafo e outro:
\setlength{\parskip}{0.2cm}  % tente também \onelineskip

% ---
% compila o indice
% ---
\makeindex
% ---

% ----
% Início do documento
% ----
\begin{document}

% Seleciona o idioma do documento (conforme pacotes do babel)
%\selectlanguage{english}
\selectlanguage{brazil}

% Retira espaço extra obsoleto entre as frases.
\frenchspacing

% ----------------------------------------------------------
% ELEMENTOS PRÉ-TEXTUAIS
% ----------------------------------------------------------
\pretextual

% ---
% Capa
% ---
\imprimircapa
% ---

% ---
% Folha de rosto
% (o * indica que haverá a ficha bibliográfica)
% ---
\imprimirfolhaderosto
% ---

% ---
% Anverso da folha de rosto:
% ---

% \begin{fichacatalografica}
%   \vspace*{15cm} % Posição vertical
%   \hrule % Linha horizontal
%   \begin{center} % Minipage Centralizado
%     \begin{minipage}[c]{12.5cm} % Largura
%       \imprimirautor
%       \hspace{0.5cm} \imprimirtitulo / \imprimirautor. --
%       \imprimirlocal, \imprimirdata-
%       \hspace{0.5cm} \pageref{LastPage} p. : il.(alguma color.); 30 cm.\\
%       \hspace{0.5cm} \imprimirorientadorRotulo \imprimirorientador\\
%       \hspace{0.5cm}
%       \parbox[t]{\textwidth}{\imprimirtipotrabalho~--~\imprimirinstituicao,
%         \imprimirdata.}\\
%       \hspace{0.5cm}
%       1. LoRaWAN.
%       2. LoRa Server.
%       I. Orientador.
%       II. Universidade xxx.
%       III. Faculdade de xxx.
%       IV. Título\\
%       \hspace{8.75cm} CDU 02:141:005.7\\
%     \end{minipage}
%   \end{center}
%   \hrule
% \end{fichacatalografica}

\begin{folhadeaprovacao}

  \begin{center}

    {\ABNTEXchapterfont\large\textsc{\imprimirautor}}

    {\ABNTEXchapterfont\Large\bfseries\imprimirtitulo}

  \end{center}

  \vspace{1cm}

  \hspace{.45\textwidth} \begin{minipage}{.45\textwidth}

    \imprimirpreambulo

  \end{minipage}

  \vspace{1cm}

  Trabalho aprovado. Uberlândia, \today

  %%%%%%%%%%%%%%%%%%%%%%%%%%

  %Assinaturas

  %%%%%%%%%%%%%%%%%%%%%%%%%%%%%%%%%%%%%%%%%%%%%%
  \assinatura{\textbf{\imprimirorientador} \\ Orientador}
  \assinatura{\textbf{Dr. Gilberto Arantes Carrijo} \\ Convidado 1}
  \assinatura{\textbf{Dr. Alan Petrônio Pinheiro} \\ Convidado 2}
  \assinatura{\textbf{Dr. Éderson Rosa da Silva} \\ Convidado 3}
  %%%%%%%%%%%%%%%%%%%%%%%%%%%%%%%%%%%%%%%%%%%%%%%%%%%

  %%%%%%%%%%%%%%%%%%%%%%%%%%%%%%%%%%%%%%%%%%%%%%%%%%%
  \begin{center}
    \vfill
    {\large\imprimirlocal}
    \par
    {\large\imprimirdata}

  \end{center}
\end{folhadeaprovacao}
%%%%%%%%%%%%%%%%%%%%%%%%%%%%%%%%%

%Fim da folha de aprovação
%%%%%%%%%%%%%%%%%%%%%%%%%%%%%%%

%%%%%%%%%%%%%%%%%%%%%%%%%%%%%%%%%
% Início da dedicatória - Elemento opcional
%%%%%%%%%%%%%%%%%%%%%%%%%%%%%%%%%%%%%%%%%%%%%%%%%%%%%%%%%%%
\begin{dedicatoria}
  \vspace*{\fill}
  Este trabalho é dedicado à minha mãe, Maria Aparecida da Silva, que sempre me foi exemplo de obstinação, diligência e honradez.
  \vspace*{\fill}

\end{dedicatoria}
%%%%%%%%%%%%%%%%%%%%%%%%%%%%%%%%%%%%%%%%%%%%%%%%%%

% Fim da dedicatória
%%%%%%%%%%%%%%%%%%%%%%%%%%%%%%%%%%%%%%%%%%%%%%%%%%



% ---
% Agradecimentos
% ---
\begin{agradecimentos}
  Agradeço primeiramente a minha família por ter apoiado e viabilizado todo esse processo de aprendizado.

  Sou grato pela liberdade e confiança dispensada pelo meu orientador, professor Dr. Alan Petrônio Pinheiro.

  Aos professores que contribuíram para o cumprimento dessa jornada.

  À Universidade Federal de Uberlândia por cumprir veementemente com o seu papel de formação de cidadãos através da garantia do ensino público, gratuito e de qualidade.

  Aos colegas do laboratório de Redes de Computadores e Telecomunicações, William, Daniel e Caio por todo apoio e amizade.

  Agradeço também a todas as entidades que estiveram presentes durante minha formação, com destaques para o Diretório Acadêmico da Faculdade de Engenharia Elétrica e ao Laboratório de Automação, Sistemas Eletrônicos e Controle (LASEC) que muito me auxiliaram no desenvolvimento profissional e de liderança.


\end{agradecimentos}
% ---

%%%%%%%%%%%%%%%%%%%%%%% 
% Início da epígrafe - opcional 
%%%%%%%%%%%%%%%%%%%%%%%%%%%%%%%%%%%%%%%%%%%%%%%%%%%%%%%%%%%%%% 
\begin{epigrafe}
  \vspace*{\fill}
  \begin{flushright}
    \textit{``Messages and the corresponding signals are points in two "function spaces", and the
      modulation process is a mapping of one space into other.''\\ (Claude E. Shannon)}
  \end{flushright}
\end{epigrafe}
%%%%%%%%%%%%%%%%%%%%%%%%%%%%%%%%%%%%%%%%%%%%%%%%%%%%%%%%%%%%%%%%% 
% Fim da epígrafe - opcional 
%%%%%%%%%%%%%%%%%%%%%%%%%%%%%%%%%%%%%%%%%%%%%%%%%%%%%%%%%%%%%%

% ---
% RESUMO
% ---

% resumo na língua vernácula (obrigatório)
\setlength{\absparsep}{18pt} % ajusta o espaçamento dos parágrafos do resumo
\begin{resumo}
  % Segundo a \citeonline[3.1-3.2]{NBR6028:2003}, o resumo deve ressaltar o
  objetivo, o método, os resultados e as conclusões do documento. A ordem e a extensão
  destes itens dependem do tipo de resumo (informativo ou indicativo) e do
  tratamento que cada item recebe no documento original. O resumo deve ser
  precedido da referência do documento, com exceção do resumo inserido no
  próprio documento. (\ldots) As palavras-chave devem figurar logo abaixo do
  resumo, antecedidas da expressão Palavras-chave:, separadas entre si por
  ponto e finalizadas também por ponto.

  \noindent
  \textbf{Palavras-chaves}: latex. abntex. editoração de texto.
\end{resumo}
% ---

% ---
% inserir lista de ilustrações
% ---
\pdfbookmark[0]{\listfigurename}{lof}
\listoffigures*
\cleardoublepage
% ---

% ---
% inserir lista de tabelas
% ---
\pdfbookmark[0]{\listtablename}{lot}
\listoftables*
\cleardoublepage
% ---

% ---
% inserir lista de abreviaturas e siglas
% ---
\begin{siglas}

  \item[CLI]  \textit{Command Line Interface}
  \item[SDK]  \textit{Software Development Kit}
  \item[FSF]  \textit{Free Software Foundation}
  \item[GNU]  \textit{GNU's Not Unix}
  \item[GPLv3]  \textit{General Public License version 3}
  \item[GRC]  \textit{GNURadio Companion}
  \item[SDR]  \textit{Software-defined Radio}
  \item[DSP]  \textit{Digital Signal Processing}
  \item[ARM]  \textit{Advanced RISC Machine}
  \item[GPG]  \textit{GNU Privacy Guard}
  \item[STDIN]  \textit{Standard Input}
  \item[TTY]  \textit{TeleTYpewriter}
  \item[GUI]  \textit{Graphical User Interface}


  \item[ABP]  \textit{Activation-ByPersonalisation}
  \item[ADR] \textit{Adaptive Data Rate}
  \item[AM] \textit{Amplitude Modulation}
  \item[API] \textit{Application Programming Interface}
  \item[ASK] \textit{Amplitude-shift keying}
  \item[BW] \textit{Bandwidth}
  \item[CR] \textit{Coding Rate}
  \item[CRC] \textit{Cyclic Redundancy Check}
  \item[CSS] \textit{Chirp Spread Spectrum}
  \item[DAC] \textit{Digital-to-Analog Converter}
  \item[DR] \textit{Data Rate}
  \item[EIRP] \textit{Effective Isotropic Radiated Power}
  \item[ERP] \textit{Effective Radiated Power}
  \item[ETSI] \textit{European Telecommunications Standards Institute}
  \item[FCC] \textit{Federal Communications Commission}
  \item[FEC] \textit{Forward Error Correction}
  \item[FM] \textit{Frequency Modulation}
  \item[FPGA] \textit{Field Programmable Gate Array}
  \item[FSK] \textit{Frequency Shift Keying}
  \item[GNSS] \textit{Global Navigation Satellite System}
  \item[GPS] \textit{Global Positioning System}
  \item[gRPC] \textit{Google Remote Procedure Call}
  \item[HAL] \textit{Hardware Abstraction Layer}
  \item[IoT] \textit{Internet of Things}
  \item[ISM band] \textit{Industrial, Scientific and Medical band}
  \item[JSON] \textit{JavaScript Object Notation}
  \item[KCC] \textit{Korea Communications Commission}
  \item[LBT] \textit{Listen-Before-Talk}
  \item[LMIC] \textit{LoraMAC-in-C}
  \item[LO] \textit{Local Oscillator}
  \item[LoRa] \textit{Long Range}
  \item[LoRaWAN] \textit{Long Range Wide Area Network}
  \item[LPF] \textit{Low-Pass Filter}
  \item[LPWA] \textit{Low Power, Wide Area}
  \item[MAC] \textit{Media Access Control}
  \item[MCU] \textit{Microcontroller Unit}
  \item[MIC] \textit{Message Integrity Code}
  \item[MQTT] \textit{Message Queuing Telemetry Transport}
  \item[M2M] \textit{Machine-to-Machine}
  \item[NB-IoT] \textit{Narrowband Internet of Things}
  \item[OTAA] \textit{Over-The-Air-Activation}
  \item[PM] \textit{Phase Modulation}
  \item[PSK] \textit{Phase-shift keying}
  \item[REST] \textit{Representational State Transfer}
  \item[RF] \textit{Radiofrequency}
  \item[RSSI] \textit{Received Signal Strength Indicator}
  \item[SF] \textit{Spreading Factor}
  \item[SI] Sistema Internacional
  \item[SNR] \textit{Signal-to-Noise Ratio}
  \item[SPI] \textit{Serial Peripheral Interface}
  \item[TELEC] \textit{Telecom Engineering Center}
  \item[ToA] \textit{Time on Air}
  \item[TTN] \textit{The Things Network}
  \item[UDP] \textit{User Datagram Protocol}

\end{siglas}
% ---

% ---
% inserir lista de símbolos
% ---
\begin{simbolos}
  \item[$ Rb $] Taxa de bits
  \item[$ Rs $] Taxa de símbolos
  \item[$ dB $] Decibel
  \item[$ P $] Potência
\end{simbolos}
% ---

% ---
% inserir o sumario
% ---
\pdfbookmark[0]{\contentsname}{toc}
\tableofcontents*
\cleardoublepage
% ---


% ----------------------------------------------------------
% ELEMENTOS TEXTUAIS
% ----------------------------------------------------------
\textual

% ----------------------------------------------------------
% Introdução (exemplo de capítulo sem numeração, mas presente no Sumário)
% ----------------------------------------------------------
\chapter*[Introdução]{Introdução}
\addcontentsline{toc}{chapter}{Introdução}

----------- USAR ESSE EXEMPLO E FALAR DE GNURADIO, DESENVOLVIMENTO DE SOFTWARE, BOAS PRÁTICAS, ENGENHARIA DE SOFTWARE, CONTAINERS, LINGUAGENS DE PROGRAMAÇÃO, C++, PYTHON
----------- DOCKER, RÁDIO DEFINIDO POR SOFTWARE, SOFTWARE-DEFINED RADIO, HACKRF, MICHAEL OSSMAN, LIMESDR, ANALOG DEVICES, SIMULINK, GNURADIO-COMPANION

A Internet das Coisas (IoT - Internet of Things) é uma rede de dispositivos físicos do cotidiano que podem se comunicar, levando em consideração que o seu foco é a interconexão entre dispositivos (conceito M2M) e, não apenas, dispositivos a seres humanos. A previsão é que mais de 25 bilhões de dispositivos estarão conectados à Internet até 2020 \cite[p. 1]{sanchez2017transmission}.

However, the solutions deployed for human’s cellular communication (e.g., Global System for Mobile communications (GSM), General Packet Radio Service (GPRS), or Long Term Evolution (LTE)) present important drawbacks that make them unsuitable to be directly used by constrained IoT devices. These technologies were designed for applications with
different requirements that those needed by IoT systems. Thus, a telephony cell is designed for providing broadband services to a limited number of users; meanwhile, an IoT cell will host a massive number of devices generating sporadic transmissions of
short packets. This potentially huge population of devices gaining connectivity through a single base station raises new challenges related to signaling and traffic control [4] pegar referencia no \cite[p. 1]{sanchez2017transmission}.

\cite[p. 1]{AN1200.13}. lORA MODEM


Este documento e seu código-fonte são exemplos de referência de uso da classe
\textsf{abntex2} e do pacote \textsf{abntex2cite}. O documento
exemplifica a elaboração de relatórios técnicos e/ou científicos produzidos
conforme a ABNT NBR 10719:2015 \emph{Informação e documentação - Relatório
  técnico e/ou científico - Apresentação}.

A expressão ``Modelo canônico'' é utilizada para indicar que \abnTeX\ não é
modelo específico de nenhuma universidade ou instituição, mas que implementa tão
somente os requisitos das normas da ABNT. Uma lista completa das normas
% observadas pelo \abnTeX\ é apresentada em \citeonline{abntex2classe}.

Sinta-se convidado a participar do projeto \abnTeX! Acesse o site do projeto em
\url{http://www.abntex.net.br/}. Também fique livre para conhecer,
estudar, alterar e redistribuir o trabalho do \abnTeX, desde que os arquivos
modificados tenham seus nomes alterados e que os créditos sejam dados aos
autores originais, nos termos da ``The \LaTeX\ Project Public
License''\footnote{\url{http://www.latex-project.org/lppl.txt}}.

Encorajamos que sejam realizadas customizações específicas deste exemplo para
universidades e outras instituições --- como capas, folhas de rosto, etc.
Porém, recomendamos que ao invés de se alterar diretamente os arquivos do
\abnTeX, distribua-se arquivos com as respectivas customizações.
Isso permite que futuras versões do \abnTeX~não se tornem automaticamente
incompatíveis com as customizações promovidas. Consulte
% \citeonline{abntex2-wiki-como-customizar} para mais informações.

Este documento deve ser utilizado como complemento dos manuais do \abnTeX\
% \cite{abntex2classe,abntex2cite,abntex2cite-alf} e da classe \textsf{memoir}
% \cite{memoir}.

Equipe \abnTeX

Lauro César Araujo


% ----------------------------------------------------------
% PARTE - preparação do ambiente de desenvolvimento
% ----------------------------------------------------------
\part{Ambiente de desenvolvimento}

\chapter{Definições}

O conceito e as boas práticas relacionadas ao desenvolvimento de \textit{software} evoluiu bastante desde os primórdios da computação e algo que sempre esteve
em pauta é o problema de incompatibilidade de ambientes de desenvolvimento, o que acabou dando origem ao jargão: “na minha máquina funciona”.
Configurar o ambiente de desenvolvimento e depois ter incompatibilidade de versão de alguma biblioteca ou então não ter a facilidade de replicar o ambiente
que foi utilizado durante o desenvolvimento é um grande problema enfrentado por engenheiros e/ou programadores, principalmente em início de carreira.
Nesta parte deste trabalho será demonstrada a criação de um ambiente para desenvolvimento de aplicações de rádio definido por \textit{software} utilizando
o \textbf{GNURadio} instalado em um \textit{container} \textbf{Docker}.
Os conceitos de \textit{container}'s e do Docker com certeza já são amplamente abordados na atualidade e então, apenas alguns pontos relacionados ao tema
serão tratados.

\section*{GNURadio}

O GNURadio é um SDK (\textit{software development kit}) de um projeto \textit{OpenSource} criado com o intuito de auxiliar no desenvolvimento de aplicações de
rádio definido por \textit{software} inicialmente publicado no ano de 2001 como um pacote oficial GNU.
O GNURadio pode ser utilizado tanto por \textit{hobbystas}, como também em meio acadêmico ou para suprir necessidades do mercado no que tange comunicações sem
fio ou qualquer tipo de sistema de rádio digital do mundo real.
O GNURadio está sob uma licença pública geral — GNU GPLv3 — da \textit{Free Software Foundation} (FSF).

Por ser um projeto de \textit{software} agnóstico ao hardware das plataformas de desenvolvimento (SDR — \textit{Software-defined Radio}), o GNURadio é idealizado
para trabalhar apenas com dados digitalizados e a partir disso executar todo um processamento digital de sinais (DSP — \textit{Digital Signal Processing})
definido por aplicações escritas para receber ou enviar dados de sistemas de \textit{streaming} digital programaticamente, utilizando a linguagem de programação
Python — que é considerado mais fácil — ou C++ — para escrita de códigos de desempenho crítico — além de também ser possível através de uma interface gráfica de
usuário (\textbf{GNURadio Companion} — GRC) por diagramas de blocos.

\section*{Docker}

O Docker é uma unidade padrão de \textit{software} da empresa Docker Inc., que fornece uma camada de abstração e automação em containers, ou seja, grupos isolados de
processos do kernel Linux sendo executados e compartilhando os recursos de um mesmo host. Um bom entendimento sobre os \textit{namespaces} do  sistema operacional
Linux (\textbf{mnt}, \textbf{pid}, \textbf{net}, \textbf{ipc}, \textbf{uts}, \textbf{user} e \textbf{cgroups}) pode ajudar na compreensão de quais soluções os
containers podem oferecer.

FALAR O PORQUÊ DE USAR O DOCKER E NÃO INSTALAR DIRETAMENTE NA MÁQUINA HOST

\chapter{Criação do ambiente de desenvolvimento}

Primeiramente é necessário ter a CLI (\textit{command line interface}) do \textbf{docker engine} instalada (na máquina \textit{host}) e tal procedimento de
instalação se torna extremamente simples bastando seguir o passo-a-passo fornecido pela \href{https://docs.docker.com/engine/install/}{documentação oficial}
do Docker \cite{Docker:2020}.
Esta CLI está disponível para várias distribuições Linux na versão \textit{Server} e também para Windows e Mac na versão \textit{Desktop}. Neste trabalho será
demonstrado procedimento de instalação do \textbf{docker engine} em uma distribuição Linux, de forma que o procedimento é semelhante para demais distribuições
e arquiteturas.

\section*{Instalação do Docker Engine}

O Docker fornece pacotes prontos (\textbf{.deb} e \textbf{.rpm}) para instalação dessa CLI em distribuições Linux como \textit{CentOS}, \textit{Debian},
\textit{Fedora}, \textit{Raspbian}, \textit{Ubuntu} e derivadas (\textit{LMDE}, \textit{BunsenLabs Linux}, \textit{Kali Linux}, \textit{Kubuntu}, \textit{Lubuntu}
e \textit{Xubuntu}, por exemplo) para arquiteturas \textbf{x86\_64}/\textbf{amd64}, \textbf{ARM} (\textit{Advanced RISC Machine}) e \textbf{ARM64}/\textbf{AARCH64}.
Outra opção de instalação seria utilizando os binários pré-compilados que são fornecidos no site do Docker e essa forma
faz bastante sentido quando o sistema operacional utilizado não fizer parte de algum dentre todas as plataformas suportadas
pelo Docker.

Antes de iniciar o procedimento de instalação é necessário verificar se o sistema operacional utilizado está instalado em uma versão de arquitetura 64-bit
(\textbf{x86\_64} ou \textbf{amd64}, \textbf{armhf} e \textbf{arm64}) e não possui instalações de versões antigas do \textit{software}
(\textbf{docker}, \textbf{docker.io}, ou \textbf{docker-engine}) e, caso hajam, é necessário removê-las. No sistema operacional Ubuntu
basta utilizar o gerenciador de pacotes \textbf{apt}/\textbf{apt-get} para fazê-lo, conforme é exemplificado a seguir:

\begin{lstlisting}[language=bash]
  $ sudo apt-get remove docker-engine \
    docker docker.io containerd runc
  \end{lstlisting}

\subsection*{Instalação utilizando repositório}

Utilizar o gerenciador de pacotes presente na distribuição Linux facilita o processo de instalação e remoção de \textit{software} do sistema operacional. O primeiro
passo é atualizar a lista de pacotes e instalar algumas dependências que permitem que o gerenciador de pacotes (\textbf{apt}) use um repositório sobre HTTPS:

\begin{lstlisting}[language=bash]

  $ sudo apt-get update

  $ sudo apt-get install \
    apt-transport-https \
    ca-certificates \
    curl \
    gnupg-agent \
    software-properties-common
  \end{lstlisting}


Após isso é necessário adicionar a chave GPG (\textit{GNU Privacy Guard}) oficial do Docker através do comando:

\begin{lstlisting}[language=bash]
  $ curl -fsSL https://download.docker.com/linux/ubuntu/gpg \
    | sudo apt-key add -
  \end{lstlisting}

Para verificar se possui a chave com o \textit{fingerprint} Docker (no momento em que este texto foi escrito,
era \textbf{9DC8 5822 9FC7 DD38 854A E2D8 8D81 803C 0EBF CD88}), o usuário deve pesquisar os últimos 8 caracteres do \textit{fingerprint}. Neste caso, basta utilizar o comando:

\begin{lstlisting}[language=bash]
    $ sudo apt-key fingerprint 0EBFCD88
    \end{lstlisting}

E o retorno do terminal ficará:

\begin{lstlisting}[language=bash]
  pub   rsa4096 2017-02-22 [SCEA]
  9DC8 5822 9FC7 DD38 854A  E2D8 8D81 803C 0EBF CD88
  uid  [unknown] Docker Release (CE deb) <docker@docker.com>
  sub   rsa4096 2017-02-22 [S]
\end{lstlisting}

Feito isso, o próximo passo será configurar que o repositório mais estável ("\textit{stable}") do Docker seja indexado ao gerenciador de pactotes do sistema, o \textbf{apt}:

\begin{lstlisting}[language=bash]
  $ sudo add-apt-repository \
  "deb [arch=amd64] https://download.docker.com/linux/ubuntu \
  $(lsb_release -cs) \
  stable"
\end{lstlisting}

Finalmente, as versões mais recentes do \textbf{containerd} e da \textit{Engine} do Docker podem ser instaladas após atualizar os índices do gerenciador de pacotes \textbf{apt}.

\begin{lstlisting}[language=bash]
  $ sudo apt-get update
  $ sudo apt-get install docker-ce \
    docker-ce-cli containerd.io
\end{lstlisting}

Para verificar se a instalação foi bem-sucedida, basta executar o comando de testes a seguir que é o "\textit{hello-world}" do docker. Se tudo ocorrer bem,
o terminal responderá com uma saída semelhante ao que é mostrado na Fig. \ref{fig:docker-hello-world}.

\begin{lstlisting}[language=bash]
  $ sudo docker run hello-world
\end{lstlisting}

\begin{figure}[!htb]
  \centering
  \caption{Teste de verificação de instalação do Docker Engine }
  \includegraphics[width = \linewidth]{figures/docker-hello-world.png}
  \label{fig:docker-hello-world}
\end{figure}

Neste ponto, o Docker encontra-se devidamente instalado no sistema operacional e o usuário pode optar por executar alguns comandos de pós-instalação
para, por exemplo, possibilitar a execução do docker sem a necessidade de ter privilégios de usuário \textit{root}, iniciar o serviço docker na inicialização do
sistema operacional, usar uma \textit{engine} de armazenamento diferente da que vem como padrão de instalação (\textit{overlay2}) dentre outras coisas descritas
na seção de \href{https://docs.docker.com/engine/install/linux-postinstall/}{pós-instalação} da documentação oficial.

\section*{Criação do container}

Com o Docker instalado, para inciar o procedimento de criação do \textit{container} do ambiente de desenvolvimento é preciso executar o comando \textbf{xhost +}
para fornecer acesso a aplicações “externas” ao \textit{host} ao seu servidor gráfico (também conhecido como sessão \textbf{X} ou a tela do computador),
afinal um dos objetivos é utilizar o \textit{GNURadio Companion} que estará instalado dentro do \textit{container}.

Agora, executando o seguinte comando será criado um \textit{container} a partir de uma imagem base da distribuição \textit{Ubuntu}. A Fig.
\ref{fig:docker-run-gnuradio} exemplifica o uso dos comandos e as saídas retornadas pelo terminal e é dentro do \textit{container}
criado que será feita a instalação do GNURadio.

\begin{lstlisting}[language=bash]
$  docker run -i -t --privileged -e DISPLAY=$DISPLAY \
      -v /tmp/.X11-unix/:/tmp/.X11-unix \
      -v /dev/usb:/dev/usb \
      -v /dev/snd:/dev/snd \
      -v $HOME:/home/developer/working \
      --name gnuradio \
      ubuntu
  \end{lstlisting}

\begin{figure}[!htb]
  \centering
  \caption{Criação do \textit{container} de desenvolvimento a partir de uma imagem Ubuntu}
  \includegraphics[width = \linewidth]{figures/docker-run-gnuradio.png}
  \label{fig:docker-run-gnuradio}
\end{figure}

Para melhorar o entendimento, essa linha vai ser dividida e explicada por partes. O comando \textbf{docker run} primeiro cria uma camada de \textit{container}
gravável sobre a imagem especificada (Ubuntu) e, em seguida, o inicia usando o comando especificado. Ou seja, \textbf{docker run} é equivalente a utilizar
\textbf{docker container create} + \textbf{docker container start}. Um \textit{container} interrompido pode ser reiniciado com todas as suas alterações anteriores
intactas usando \textbf{docker start}. Agora, sobre as \textit{flags} passadas:

\begin{itemize}
  \item[$-$] \textbf{i} (\textbf{interactive}): manter o \textbf{STDIN} (\textit{standard input} — fluxo de entrada padrão) aberto, mesmo se não estiver conectado;
  \item[$-$] \textbf{t} (\textbf{tty}): alocar um pseudo-\textbf{TTY} (\textit{TeleTYpewriter} — simplesmente um terminal ao qual o usuário está conectado);
  \item[$-$] \textbf{privileged}: dar privilégios estendidos a este \textit{container} (O docker permitirá o acesso a todos os dispositivos no \textit{host},
        bem como definirá algumas configurações no AppArmor ou SELinux para permitir ao container quase todo o mesmo acesso ao \textit{host} que os processos que executam
        \textit{container}'s externos no \textit{host}.);
  \item[$-$] \textbf{e} (\textbf{env}): definir variáveis de ambiente;
  \item[$-$] \textbf{v} (\textbf{volume}): vincular a montagem de um volume dentro \textit{container};
  \item[$-$] \textbf{name}: o nome que se dará ao \textit{container} criado.
\end{itemize}

Resumindo, foi utilizada uma linha de comando para criar um \textit{container}, o qual foi nomeado \textbf{gnuradio}, a partir de uma imagem base (Ubuntu) com
privilégios para acessar recursos da máquina (o \textit{host}), onde mais precisamente será possível utilizar os recursos de áudio, da porta USB e da
interface visual com uma vinculação de volumes montados no \textit{container} a partir da máquina \textit{host}.

\begin{figure}[!htb]
  \centering
  \caption{Detalhamento de informações sobre \textit{container}’s Docker}
  \includegraphics[width = \linewidth]{figures/docker-inspect.png}
  \label{fig:docker-inspect}
\end{figure}

Algo importante a se notar na Fig. \ref{fig:docker-run-gnuradio} é que depois da criação do \textit{container} o terminal “mudou” do usuário \textbf{jefferson} para \textbf{root} e do
\textit{hostname} \textbf{jefferson} para \textbf{142104d3a3fb}. Isso ocorre porque a partir desse momento o que aparece na tela é o terminal visto de "dentro"
do \textbf{container}, ou seja com o usuário \textbf{root} do \textit{container} de \textit{hostname} \textbf{142104d3a3fb}.

Em um outro terminal é possível obter mais informações como, por exemplo, quais \textit{container}'s estão sendo executados no momento ou inspecionar algum
\textit{container} específico em busca de maiores detalhes, como é mostrado na Fig. \ref{fig:docker-inspect}.

\section*{Instalação do GNURadio}

Até o momento apenas foi feito o provisionamento de um \textit{container} docker executando uma imagem Ubuntu e então o GNURadio será instalado utilizando
o gerenciador de pacotes \textbf{apt} disponível na distribuição Ubuntu deste \textit{container}. É recomendável atualizar a lista dos repositórios e verificar
as depedências para esta instalação bastando executar os seguintes comandos no terminal do \textit{container} criado:

\begin{lstlisting}[language=bash]
  $ apt-get update && apt-get upgrade
  $ apt-get install libgmp-dev gir1.2-gtk-3.0 xauth
  $ apt-get install gnuradio
\end{lstlisting}

A partir desse ponto, o GNURadio estará instalado no \textit{container} docker e é possível verificar a versão, o local de instalação e os
componentes já habilitados através dos seguintes comandos que foram exemplifiados na Fig. \ref{fig:gnuradio-config-info}:

\begin{lstlisting}[language=bash]
  $ gnuradio-config-info --version
  $ gnuradio-config-info --prefix
  $ gnuradio-config-info --enabled-components
\end{lstlisting}

\begin{figure}[!htb]
  \centering
  \caption{Conferência de versão, \textit{path} e componentes habilitados do GNURadio}
  \includegraphics[width = \linewidth]{figures/gnuradio-config-info.png}
  \label{fig:gnuradio-config-info}
\end{figure}

Conforme ilustrado na Fig. \ref{fig:gnuradio-companion}, após todos os procedimentos descritos neste capítulo, o GNURadio na versão \textbf{3.8.1.0} estará
disponível para o desenvolvimento de aplicações de rádio definido por \textit{software} e de processamento digital de sinais utilizando as bibliotecas
que foram instaladas no \textit{container} ou utilizando a GUI — \textit{Graphical User Interface}, interface gráfica de usuário —
através do comando \textbf{gnuradio-companion} (GRC).

\begin{figure}[!htb]
  \centering
  \caption{GUI do GNURadio Companion}
  \includegraphics[width = \linewidth]{figures/gnuradio-companion.png}
  \label{fig:gnuradio-companion}
\end{figure}

\newpage
\section*{Criação da imagem Docker}

CRIAR A IMAGEM DOCKER

DAR UMA TAG PARA A IMAGEM DOCKER

ENVIAR A IMAGEM DOCKER PARA O REGISTRY NA DOCKER HUB

% ----------------------------------------------------------
% Parte de resultados
% ----------------------------------------------------------
\part{Resultados}

% ---
% Capitulo de revisão de literatura
% ---
\chapter{Lorem ipsum dolor sit amet}

% ---
\section{Aliquam vestibulum fringilla lorem}
% ---

\lipsum[1]

\lipsum[2-3]

% ---
% Finaliza a parte no bookmark do PDF
% para que se inicie o bookmark na raiz
% e adiciona espaço de parte no Sumário
% ---
\phantompart

% ---
% Considerações Finais
% ---
\chapter{Considerações Finais}
% ---

\lipsum[31-33]

% ----------------------------------------------------------
% ELEMENTOS PÓS-TEXTUAIS
% ----------------------------------------------------------
\postextual

% ----------------------------------------------------------
% Referências bibliográficas
% ----------------------------------------------------------
\bibliography{abntex2-modelo-references}

% ----------------------------------------------------------
% Glossário
% ----------------------------------------------------------
%
% Consulte o manual da classe abntex2 para orientações sobre o glossário.
%
%\glossary

% ----------------------------------------------------------
% Apêndices
% ----------------------------------------------------------

% ---
% Inicia os apêndices
% ---
\begin{apendicesenv}

  % Imprime uma página indicando o início dos apêndices
  % \partapendices

  % ----------------------------------------------------------
  % \chapter{Quisque libero justo}
  % ----------------------------------------------------------

  \lipsum[50]

  % ----------------------------------------------------------
  \chapter{Nullam elementum urna vel imperdiet sodales elit ipsum pharetra ligula
    ac pretium ante justo a nulla curabitur tristique arcu eu metus}
  % ----------------------------------------------------------
  \lipsum[55-57]

\end{apendicesenv}
% ---


% ----------------------------------------------------------
% Anexos
% ----------------------------------------------------------

% ---
% Inicia os anexos
% ---
\begin{anexosenv}

  % Imprime uma página indicando o início dos anexos
  % \partanexos
  % ---
  % \chapter{Morbi ultrices rutrum lorem.}
  % ---
  % \lipsum[30]

  % ---
  \chapter{Cras non urna sed feugiat cum sociis natoque penatibus et magnis dis
    parturient montes nascetur ridiculus mus}
  % ---

  \lipsum[31]

  % ---
  \chapter{Fusce facilisis lacinia dui}
  % ---

  \lipsum[32]

\end{anexosenv}

%---------------------------------------------------------------------
% INDICE REMISSIVO
%---------------------------------------------------------------------

\phantompart

\printindex

%---------------------------------------------------------------------
% Formulário de Identificação (opcional)
%---------------------------------------------------------------------
\chapter*[Formulário de Identificação]{Formulário de Identificação}
\addcontentsline{toc}{chapter}{Exemplo de Formulário de Identificação}
\label{formulado-identificacao}

Exemplo de Formulário de Identificação, compatível com o Anexo A (informativo)
da ABNT NBR 10719:2015. Este formulário não é um anexo. Conforme definido na
norma, ele é o último elemento pós-textual e opcional do relatório.

\bigskip

\begin{tabular}{|p{8cm}|p{5cm}|}
  \hline
  \multicolumn{2}{|c|}{\textbf{\large Dados do Relatório Técnico e/ou científico}}               \\
  \hline
  \multirow{3}{8cm}[24pt]{Título e subtítulo} & Classificação de segurança                       \\
                                              &                                                  \\
  \cline{2-2}
                                              & No.                                              \\
                                              &                                                  \\

  \hline
  Tipo de relatório                           & Data                                             \\
  \hline
  Título do projeto/programa/plano            & No.                                              \\
  \hline
  \multicolumn{2}{|l|}{Autor(es)}                                                                \\
  \hline
  \multicolumn{2}{|l|}{Instituição executora e endereço completo}                                \\
  \hline
  \multicolumn{2}{|l|}{Instituição patrocinadora e endereço completo}                            \\
  \hline
  \multicolumn{2}{|l|}{Resumo}                                                                   \\[3cm]
  \hline
  \multicolumn{2}{|l|}{Palavras-chave/descritores}                                               \\
  \hline
  \multicolumn{2}{|l|}{
    Edição \hfill No. de páginas \hfill No. do volume \hfill Nº de classificação \phantom{XXXX}} \\
  \hline
  \multicolumn{2}{|l|}{
    ISSN \hfill \hfill Tiragem \hfill Preço \phantom{XXXXXXXX}}                                  \\
  \hline
  \multicolumn{2}{|l|}{Distribuidor}                                                             \\
  \hline
  \multicolumn{2}{|l|}{Observações/notas}                                                        \\[3cm]
  \hline
\end{tabular}

\end{document}
